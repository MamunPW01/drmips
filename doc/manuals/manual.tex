% Short manual that explains the end user how to use the simulator.

\documentclass[11pt,a4paper,twoside,titlepage]{article}

\usepackage[utf8x]{inputenc}
\usepackage[english]{babel}
\usepackage[T1]{fontenc}
\usepackage{amsmath}
\usepackage{amsfonts}
\usepackage{amssymb}
\usepackage{graphicx}
\usepackage{listings}
\usepackage{hyperref}
\usepackage{color}
\usepackage{siunitx}

\newcommand{\Author}{Bruno Nova}
\newcommand{\Title}{DrMIPS -- User Manual}

\author{Bruno Nova}
\title{\Title}
\hypersetup{pdfauthor={Bruno Nova},pdftitle={\Title}}
\definecolor{cloudwhite}{cmyk}{0,0,0,0.025}
\graphicspath{{figures/}}

\lstset{
	extendedchars=\true,
	inputencoding=utf8x,
	literate={ç}{{\c{c}}}1,
	basicstyle=\footnotesize\ttfamily,
	keywordstyle=\bfseries,
	numbers=left,
	numberstyle=\scriptsize\texttt,
	stepnumber=1,
	numbersep=8pt,
	frame=tblr,
	float=htb,
	aboveskip=8mm,
	belowskip=4mm,
	backgroundcolor=\color{cloudwhite},
	showspaces=false,
	showstringspaces=false,
	showtabs=false,
	tabsize=2,	
	captionpos=b,
	breaklines=true,
	breakatwhitespace=false,
	escapeinside={\%*}{*)},
	morekeywords={*,var,template,new}
}

\definecolor{orange}{RGB}{255,128,0}
\definecolor{cyan2}{RGB}{0,170,230}

\newcommand{\menupath}[1]{\textbf{\emph{#1}}}



\begin{document}

\maketitle
\tableofcontents
\newpage

\section{User Interface Composition}

The graphical user interface of the DrMIPS simulator is composed by the menu 
bar, the toolbar and the area where the main contents of the simulator 
are displayed.
The icons in the toolbar are shortcuts for some frequently used actions in the
menus. You can hover the mouse cursor over each icon to find out what it does.

The interface is displayed with a light theme by default. But you can switch
to the dark theme by selecting \menupath{View > Dark theme} in the menu.
DrMIPS supports multiple languages, so these names may be different.
You can choose another language in the \menupath{View > Language} menu.

The main contents of the simulator are split in tabs by default. Each tab can
be positioned on the left side of the window or on the right side. 
You can move a tab to the other side by right-clicking on it and selecting 
\menupath{Switch side} in the menu that appears.
If you would prefer to see the contents split in windows instead, you can do
that by selecting \menupath{View > Internal windows} in the menu.
The position and size of the tabs/windows are remembered on exit.

The tabs or windows displayed are:
\begin{itemize}
	\item \textbf{Code}: the code editor, where you can create or edit an
		assembly program to be executed by the simulator.
	\item \textbf{Assembled}: after the code is assembled successfully, this
		displays the resulting machine instructions.
	\item \textbf{Registers}: lists all the registers and their values.
	\item \textbf{Data memory}: displays all the values in the data memory.
	\item \textbf{Datapath}: the graphical representation of the datapath,
		and its state, is displayed here.
\end{itemize}


\section{Tabs/Windows}

\subsection{Code}

This is the code editor, where you can write an assembly program.
The editor displays the numbers of the lines and highlights the valid syntax.

You can press \emph{Ctrl+Space} to auto-complete the keyword you are writing.
Doing so displays a list of instructions, pseudo-instructions, directives and
labels that can complete that keyword.
It also displays a window that shows what the selected keyword in the list does
and how it is used.
Pressing \emph{Ctrl+Space} on an empty space will list all available
instructions, pseudo-instructions, directives and labels.
You can also view this information in the 
\menupath{CPU > Supported instructions} menu.

You can undo/redo changes, cut/copy/paste text and find/replace words in the
code. These actions can be accessed through the \menupath{Edit} menu or by
right-clicking in the code editor.
The code can be saved to or read from a file. These actions are available in
the \menupath{File} menu.

After writing your assembly program, you must assemble it.
This is done by selecting \menupath{Execute > Assemble} from the menu or by
pressing the respective button in the toolbar.
If the program has errors, a message box will indicate the first error and an
exclamation icon will appear next to the numbers of the lines that have an
error.
You can hover the mouse cursor over those icons to discover what is the error.
If the program is correct, no error message is displayed and you can then
proceed to execute it.


\subsection{Assembled} \label{sec:assembled}

After the code is successfully assembled, the resulting machine code 
instructions will be displayed here in a table.

Each line in the table corresponds to an ``assembled'' instruction, and it
contains its address, machine code and original instruction.
The instruction currently being executed is highlighted.

If simulating a pipeline processor, all the instructions that are in the
pipeline are highlighted in different colors, each one representing a
different stage.
The different colors mean:
\begin{itemize}
	\item \textbf{\textcolor{cyan2}{Blue}}: \textbf{I}nstruction 
		\textbf{F}etch (\textbf{IF}) stage.
	\item \textbf{\textcolor{green}{Green}}: \textbf{I}nstruction 
		\textbf{D}ecode (\textbf{ID}) stage.
	\item \textbf{\textcolor{magenta}{Magenta}}: \textbf{Ex}ecute 
		(\textbf{EX}) stage.
	\item \textbf{\textcolor{orange}{Orange}}: \textbf{Mem}ory access
		(\textbf{MEM}) stage.
	\item \textbf{\textcolor{red}{Red}}: \textbf{W}rite \textbf{B}ack 
		(\textbf{WB}) stage.
\end{itemize}

Hovering the mouse cursor over an instruction in the table will present a
tooltip. This tooltip displays the type of the instruction and the values
of its fields. \footnote{In the Android version, press the instruction to
see its tooltip.}

The values are shown in decimal format by default.
You can change that format to binary or hexadecimal by using the combo box
at the bottom of the window/tab.
This combo box is available in all tabs/windows, except in the code tab/window.

To control the simulation, you can the \menupath{Execute} menu or the toolbar.
Press \menupath{Step} to execute one instruction, \menupath{Back step}
to revert one instruction, \menupath{Run} to execute the entire program and
\menupath{Restart} to revert to the first instruction.


\subsection{Registers and Data Memory}

These two tabs/windows are very similar.
The registers tab/window displays the values of the registers and the program
counter, while the data memory tab/window displays the values in the data
memory.

The values that are currently being accessed are highlighted in different 
colors. The colors mean:
\begin{itemize}
	\item \textbf{\textcolor{green}{Green}}: the register/address is being 
		read by the register bank/data memory.
	\item \textbf{\textcolor{red}{Red}}: the register/address is being written 
		to the register bank/data memory.
	\item \textbf{\textcolor{orange}{Orange}}: the register/address is being 
		read and written at the same time in the register bank/data memory.
\end{itemize}

You can edit the value of any register or memory address by double-clicking
it in the respective table. This includes the program counter.
Constant registers (like the register \verb+$zero+) cannot be edited.
\footnote{In the Android version, long-press the register/address to edit it.}

By default, the values of the registers and data memory are reset every time
the code is assembled.
If you don't want that to happen, uncheck 
\menupath{Execute > Reset data before assembling} in the menu.


\subsection{Datapath}

The graphical representation of the processor's datapath is displayed here.
This is where you can see how the CPU works internally.

DrMIPS can simulate several different unicycle and pipeline datapaths.
The name of the datapath currently being used is shown in the bottom of this
tab/window, and you can choose another datapath by selecting 
\menupath{CPU > Load} from the menu.
Note that different datapaths may support different instructions.

The datapaths provided by default are:
\begin{itemize}
	\item \textbf{Unicycle datapaths}
	\begin{itemize}
		\item \textbf{unicycle.cpu}: The default unicycle datapath.
		\item \textbf{unicycle-no-jump.cpu}: Simpler variant of the unicycle 
			datapath that doesn't support the \verb+j+~(\emph{jump}) 
			instruction.
		\item \textbf{unicycle-no-jump-branch.cpu}: An even simpler variant 
			of the unicycle datapath that doesn't support \emph{jumps} nor
			\emph{branches}.
		\item \textbf{unicycle-extended.cpu}: A variant of the unicycle 
			datapath that supports some additional instructions, like 
			multiplications and divisions.
	\end{itemize}
	
	\item \textbf{Pipeline datapaths}
	\begin{itemize}
		\item \textbf{pipeline.cpu}: The default pipeline datapath, which
			implements hazard detection. The pipeline datapath doesn't
			support the \verb+j+~(\emph{jump}) instruction.
		\item \textbf{pipeline-only-forwarding.cpu}: Variant of the pipeline
			datapath that, in terms of hazard detection, only implements
			data forwarding (giving wrong results).
		\item \textbf{pipeline-no-hazard-detection.cpu}: A variant of the
			pipeline datapath that doesn't implement any form of hazard 
			detection at all (giving wrong results).
		\item \textbf{pipeline-extended.cpu}: A variant that supports some
			additional instructions, like \emph{unicycle-extended.cpu}.
	\end{itemize}
\end{itemize}

On the top of the tab/window, the instruction or instructions currently 
being executed are displayed. They are highlighted with the same colors
explained in subsection~\ref{sec:assembled}.

The datapath is displayed below of the instruction.
The components are represented by rectangles or squares and the wires by
lines that end with arrows.
Relevant wires that are in the control path are shown in blue.
Wires that are considered irrelevant in the current clock cycle are shown
in gray.
A wire is considered irrelevant if it's a control signal set to \verb+0+,
if its value is ignored by a component, if it's the output of the
\emph{hazard detection unit} and a stall isn't occurring, etc.

The values at some inputs and outputs of some important components are
show in the datapath as small ``data tips'' with yellow background.
You can hover the mouse cursor over these ``data tips'' to find out what
is the identifier of the input/output.

By hovering the mouse cursor over a component, a tooltip with some details
about it will be displayed \footnote{In the Android version, press the 
component to see its tooltip.}.
The tooltip presents the name of the component, a description of what it
does and the values at all the inputs and outputs.

You can hide the control path by unchecking 
\menupath{Datapath > Control path} in the menu.
You can hide the arrows in the end of the wires by unchecking
\menupath{Datapath > Arrows in wires}.
And you can hide the ``data tips'' by unchecking
\menupath{Datapath > Overlayed data} in the menu.

\bigskip

The datapath can also be displayed in a ``performance mode''. You can
switch to this mode by selecting \menupath{Datapath > Performance mode}
in the menu.
In this mode, the performance of the processor is simulated, and its
critical path is displayed in red.

Each component has a latency, which can be consulted in its tooltip.
The tooltip also shows the accumulated latencies at the inputs (the time it
takes for that input to receive the correct value after the clock transition)
and at the outputs (the time it takes for the component to generate the
correct value for the output).

The latencies of the components can be edited by double-clicking them while
in performance mode \footnote{In the Android version, long-press the 
component while in performance mode to edit its latency.}.
Additionally, you can select \menupath{Datapath > Restore latencies} in the
menu to restore the latencies of all components to their original values,
and \menupath{Datapath > Remove latencies} to set all latencies to \verb+0+.

You can also view some statistics about the simulation, like clock frequency
and CPI (\emph{Cycles Per Instruction}), by selecting 
\menupath{Datapath > Statistics} in the menu.


\end{document}
