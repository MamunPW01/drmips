% Short manual that explains the end user how to use the simulator (pt_PT).

\documentclass[11pt,a4paper,twoside,titlepage]{article}

\usepackage[utf8x]{inputenc}
\usepackage[portuguese]{babel}
\usepackage[T1]{fontenc}
\usepackage{amsmath}
\usepackage{amsfonts}
\usepackage{amssymb}
\usepackage{graphicx}
\usepackage{listings}
\usepackage{hyperref}
\usepackage{color}
\usepackage{siunitx}

\newcommand{\Author}{Bruno Nova}
\newcommand{\Title}{DrMIPS -- Manual de Utilizador}

\author{Bruno Nova}
\title{\Title}
\hypersetup{pdfauthor={Bruno Nova},pdftitle={\Title}}
\definecolor{cloudwhite}{cmyk}{0,0,0,0.025}
\graphicspath{{figures/}}

\lstset{
	extendedchars=\true,
	inputencoding=utf8x,
	literate={ç}{{\c{c}}}1,
	basicstyle=\footnotesize\ttfamily,
	keywordstyle=\bfseries,
	numbers=left,
	numberstyle=\scriptsize\texttt,
	stepnumber=1,
	numbersep=8pt,
	frame=tblr,
	float=htb,
	aboveskip=8mm,
	belowskip=4mm,
	backgroundcolor=\color{cloudwhite},
	showspaces=false,
	showstringspaces=false,
	showtabs=false,
	tabsize=2,	
	captionpos=b,
	breaklines=true,
	breakatwhitespace=false,
	escapeinside={\%*}{*)},
	morekeywords={*,var,template,new}
}

\definecolor{orange}{RGB}{255,128,0}
\definecolor{cyan2}{RGB}{0,170,230}

\newcommand{\menupath}[1]{\textbf{\emph{#1}}}



\begin{document}

\maketitle
\tableofcontents
\newpage

\section{Composição da Interface Gráfica}

A interface gráfica do simulador DrMIPS é composta pela barra de menus, pela
barra de ferramentas e pela área onde os conteúdos principais do simulador são
mostrados.
Os ícones na barra de ferramentas são atalhos para algumas acções usadas
frequentemente nos menus. Pode pairar o cursor do rato sobre cada ícone para
descobrir o que ele faz.

A interface é mostrada com um tema claro, por omissão. Mas pode mudar para o
tema escuro ao seleccionar \menupath{Ver > Tema escuro} no menu.
O DrMIPS suporta múltiplos idiomas, portanto estes nomes podem ser diferentes.
Pode escolher outro idioma no menu \menupath{View > Idioma}.

Os conteúdos principais do simulador são divididos em separadores, por omissão.
Cada separador pode ser posicionado no lado esquerdo na janela ou no lado 
direito.
Pode mover um separador para o outro lado se clicar com o botão direito do rato
no separador e seleccionar \menupath{Mudar de lado} no menu que aparece.
Se, em vez disso, preferir ver os conteúdos divididos em janelas, pode fazê-lo
seleccionando \menupath{Ver > Janelas internas} no menu.
As posições e tamanhos dos separadores/janelas são recordadas ao sair.

Os separadores ou janelas mostrados são:
\begin{itemize}
	\item \textbf{Código}: o editor de código, onde pode criar ou editar um
		programa em \emph{assembly} para ser executado pelo simulador.
	\item \textbf{Assemblado}: após o código ser assemblado com sucesso, 
		este mostra as instruções máquina resultantes.
	\item \textbf{Registos}: lista todos os registos e seus valores.
	\item \textbf{Memória de dados}: mostra todos os valores na memória de 
		dados.
	\item \textbf{Caminho de dados}: a representação gráfica do caminho de
		dados, e o seu estado, é mostrada aqui.
\end{itemize}


\section{Separadores/Janelas}

\subsection{Código}

Este é o editor de código, onde pode escrever um programa em \emph{assembly}.
O editor mostra os números das linhas e realça a sintaxe válida.

Pode desfazer/refazer alterações, cortar/copiar/colar texto e 
procurar/substituir palavras no código. Estas acções podem ser acedidas pelo
menu \menupath{Editar} ou pelo clique com o botão direito do rato no editor.
O código pode ser gravado para ou lido de um ficheiro. Estas acções estão
disponíveis no menu \menupath{Ficheiro}.

Pode premir \emph{Ctrl+Espaço} para auto-completar a palavra que está a escrever.
Ao fazer isso será mostrada uma lista de instruções, pseudo-instruções,
directivas e etiquetas que podem completar essa palavra.
Também será mostrada uma janela que explica o que a palavra seleccionada na lista
faz e como é usada.
Ao premir \emph{Ctrl+Espaço} num espaço vazio serão listadas todas as instruções,
pseudo-instruções, directivas e etiquetas disponíveis.
Também pode ver esta informação no menu \menupath{CPU > Instruções suportadas}.

Após escrever um programa em \emph{assembly}, terá de o assemblar.
Isto feito seleccionando \menupath{Executar > Assemblar} no menu or pressionando
o respectivo botão na barra de ferramentas.
Se o programa tiver erros, uma mensagem irá indicar o primeiro erro e um ícone de
exclamação irá aparecer ao lado dos números das linhas que têm um erro.
Pode pairar o cursor do rato sobre estes ícones para descobrir qual é o erro.
Se o programa estiver correcto, nenhuma mensagem será mostrada e pode prosseguir
para o executar.


\subsection{Assemblado} \label{sec:assembled}

Após o programa ser assemblado com sucesso, as instruções em código máquina
resultantes serão mostradas aqui numa tabela.

Cada linha da tabela corresponde a uma instrução ``assemblada'', contendo o seu
endereço, código máquina e instrução original.
A instrução a ser executada no momento é realçada.

Se estiver a simular um processador pipeline, todas as instruções que estão no
pipeline são realçadas com diferentes cores, cada uma representando uma uma etapa
diferente.
As diferentes cores significam:
\begin{itemize}
	\item \textbf{\textcolor{cyan2}{Azul}}: Etapa \textbf{I}nstruction 
		\textbf{F}etch (\textbf{IF}).
	\item \textbf{\textcolor{green}{Verde}}: Etapa \textbf{I}nstruction 
		\textbf{D}ecode (\textbf{ID}).
	\item \textbf{\textcolor{magenta}{Magenta}}: Etapa \textbf{Ex}ecute 
		(\textbf{EX}).
	\item \textbf{\textcolor{orange}{Laranja}}: Etapa \textbf{Mem}ory access
		(\textbf{MEM}).
	\item \textbf{\textcolor{red}{Vermelho}}: Etapa \textbf{W}rite \textbf{B}ack 
		(\textbf{WB}).
\end{itemize}

Ao pairar o cursor do rato sobre uma instrução na tabela irá exibir uma dica.
Esta dica mostra o tipo da instrução e os valores dos seus campos. \footnote{Na
versão para Android, toque na instrução para ver a sua dica.}

Os valores são mostrados em formato decimal, por omissão.
Pode alterar esse formato para binário ou hexadecimal usando a caixa de 
combinação no fundo do separador/janela.
Esta caixa de combinação está disponível em todos os separadores/janelas, 
excepto no separador/janela de código.

Para controlar a simulação, pode usar o menu \menupath{Executar} ou a barra de
ferramentas. Prima \menupath{Passo} para executar uma instrução, 
\menupath{Passo atrás} para reverter uma instrução, \menupath{Executar} para
executar o programa inteiro e \menupath{Reiniciar} para reverter para a primeira
instrução.


\subsection{Registers and Data Memory}

These two tabs/windows are very similar.
The registers tab/window displays the values of the registers and the program
counter, while the data memory tab/window displays the values in the data
memory.

The values that are currently being accessed are highlighted in different 
colors. The colors mean:
\begin{itemize}
	\item \textbf{\textcolor{green}{Green}}: the register/address is being 
		read by the register bank/data memory.
	\item \textbf{\textcolor{red}{Red}}: the register/address is being written 
		to the register bank/data memory.
	\item \textbf{\textcolor{orange}{Orange}}: the register/address is being 
		read and written at the same time in the register bank/data memory.
\end{itemize}

You can edit the value of any register or memory address by double-clicking
it in the respective table. This includes the program counter.
Constant registers (like the register \verb+$zero+) cannot be edited.
\footnote{In the Android version, long-press the register/address to edit it.}

By default, the values of the registers and data memory are reset every time
the code is assembled.
If you don't want that to happen, uncheck 
\menupath{Execute > Reset data before assembling} in the menu.


\subsection{Datapath}

The graphical representation of the processor's datapath is displayed here.
This is where you can see how the CPU works internally.

DrMIPS can simulate several different unicycle and pipeline datapaths.
The name of the datapath currently being used is shown in the bottom of this
tab/window, and you can choose another datapath by selecting 
\menupath{CPU > Load} from the menu.
Note that different datapaths may support different instructions.

The datapaths provided by default are:
\begin{itemize}
	\item \textbf{Unicycle datapaths}
	\begin{itemize}
		\item \textbf{unicycle.cpu}: The default unicycle datapath.
		\item \textbf{unicycle-no-jump.cpu}: Simpler variant of the unicycle 
			datapath that doesn't support the \verb+j+~(\emph{jump}) 
			instruction.
		\item \textbf{unicycle-no-jump-branch.cpu}: An even simpler variant 
			of the unicycle datapath that doesn't support \emph{jumps} nor
			\emph{branches}.
		\item \textbf{unicycle-extended.cpu}: A variant of the unicycle 
			datapath that supports some additional instructions, like 
			multiplications and divisions.
	\end{itemize}
	
	\item \textbf{Pipeline datapaths}
	\begin{itemize}
		\item \textbf{pipeline.cpu}: The default pipeline datapath, which
			implements hazard detection. The pipeline datapath doesn't
			support the \verb+j+~(\emph{jump}) instruction.
		\item \textbf{pipeline-only-forwarding.cpu}: Variant of the pipeline
			datapath that, in terms of hazard detection, only implements
			data forwarding (giving wrong results).
		\item \textbf{pipeline-no-hazard-detection.cpu}: A variant of the
			pipeline datapath that doesn't implement any form of hazard 
			detection at all (giving wrong results).
		\item \textbf{pipeline-extended.cpu}: A variant that supports some
			additional instructions, like \emph{unicycle-extended.cpu}.
	\end{itemize}
\end{itemize}

On the top of the tab/window, the instruction or instructions currently 
being executed are displayed. They are highlighted with the same colors
explained in subsection~\ref{sec:assembled}.

The datapath is displayed below of the instruction.
The components are represented by rectangles or squares and the wires by
lines that end with arrows.
Relevant wires that are in the control path are shown in blue.
Wires that are considered irrelevant in the current clock cycle are shown
in gray.
A wire is considered irrelevant if it's a control signal set to \verb+0+,
if its value is ignored by a component, if it's the output of the
\emph{hazard detection unit} and a stall isn't occurring, etc.

The values at some inputs and outputs of some important components are
show in the datapath as small ``data tips'' with yellow background.
You can hover the mouse cursor over these ``data tips'' to find out what
is the identifier of the input/output.

By hovering the mouse cursor over a component, a tooltip with some details
about it will be displayed \footnote{In the Android version, press the 
component to see its tooltip.}.
The tooltip presents the name of the component, a description of what it
does and the values at all the inputs and outputs.

You can hide the control path by unchecking 
\menupath{Datapath > Control path} in the menu.
You can hide the arrows in the end of the wires by unchecking
\menupath{Datapath > Arrows in wires}.
And you can hide the ``data tips'' by unchecking
\menupath{Datapath > Overlayed data} in the menu.

\bigskip

The datapath can also be displayed in a ``performance mode''. You can
switch to this mode by selecting \menupath{Datapath > Performance mode}
in the menu.
In this mode, the performance of the processor is simulated, and its
critical path is displayed in red.

Each component has a latency, which can be consulted in its tooltip.
The tooltip also shows the accumulated latencies at the inputs (the time it
takes for that input to receive the correct value after the clock transition)
and at the outputs (the time it takes for the component to generate the
correct value for the output).

The latencies of the components can be edited by double-clicking them while
in performance mode \footnote{In the Android version, long-press the 
component while in performance mode to edit its latency.}.
Additionally, you can select \menupath{Datapath > Restore latencies} in the
menu to restore the latencies of all components to their original values,
and \menupath{Datapath > Remove latencies} to set all latencies to \verb+0+.

You can also view some statistics about the simulation, like clock frequency
and CPI (\emph{Cycles Per Instruction}), by selecting 
\menupath{Datapath > Statistics} in the menu.


\end{document}
